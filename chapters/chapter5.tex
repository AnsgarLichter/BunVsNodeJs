% !TeX root = ..\main.tex
\npchapter{Schlussbetrachtung} \label{finalThoughts}
Im letzten Kapitel dieser Arbeit werden die Ergebnisse der vorangegangenen Kapitel reflektiert. Darauffolgend wird ein Ausblick auf die Zukunft von Bun gegeben.

\section{Fazit} \label{sec:finalThoughts-conclusion}
Diese Arbeit beschäftigt sich mit der Evaluierung von Bun im Vergleich zu Node.js. Hierzu existieren drei Leitfragen. Um diese zu beantworten, wurden die Performance von Bun im Vergleich zu Node.js und die Kompatibilität mit bestehenden Projekten auf der Basis von Node.js analysiert. \newline
Mehrere Benchmarks wurden durchgeführt, um die erste Leitfrage ist ``Welche konkreten Leistungsverbesserungen können in Bun 1.0 im Vergleich zu Node.js festgestellt werden, und wie lassen sie sich quantifizieren?'' zu beantworten. Die Ergebnisse zeigen, dass Bun in allen betrachteten Szenarien deutlich besser performt als Node.js. Es können in derselben Zeit deutlich mehr HTTP-Anfragen erfolgreich bearbeitet werden. Hierbei ist die Auslastung des Prozessors und des Arbeitsspeicher geringer. Dieses Bild bestätigt sich auch bei der Ausführung rechenintensiver Anwendungen. Bun ist schneller und benötigt weniger Ressourcen. Zusammenfassend lässt sich sagen, dass die Performance von Bun grundlegend besser ist. Allerdings wurden hier native Anwendungen basierend auf den Spezifika der einzelnen Implementation verwendet. Ob die Performance beim Einsatz von Frameworks besser ist, hängt von der Implementation des Frameworks ab. Zusätzlich handelt es sich bei den betrachteten Beispielen um einfache Anwendungsszenarien.\\

\noindent
Um die zweite Leitfrage ``Inwiefern sind Node.js-Projekte kompatibel mit Bun? Wie schwierig gestaltet sich die Migration?'' zu beantworten, wurden zwei Applikationen von Node.js zu Bun als Laufzeitumgebung migriert. Diese Applikationen basieren auf den zwei am häufigsten verwendeten Backend-Frameworks für Node.js. Grundsätzlich können beide Frameworks mit Bun ausgeführt werden. Die Migration und der dadurch entstehende Aufwand ist stark abhängig von den verwendeten Modulen und der genutzten Schnittstellen der Node.js API. Bun implementiert die Node.js API noch nicht vollständig. Daher ist der Migrationsaufwand im besten Fall sehr niedrig, da Bun die verwendeten Schnittstellen unterstützt. Im schlechtesten Fall fällt der Aufwand durch Fehlersuche, Evaluierung von Alternativen und den möglichen Umbau der Applikation sehr hoch aus. Zusätzlich kann es noch zu Laufzeitfehlern kommen, die nachfolgend sehr schwierig zu analysieren sind. Es gilt anzumerken, dass Buns Bundler schafft es noch nicht schafft die Nest.js-Applikation zu minifizieren. Die Analyse dieser Arbeit beschränkt sich auf einfache, nicht produktiv verwendete Applikationen. Bei diesen kann sich der Migrationsaufwand anders verhalten. Darüber hinaus wurden der integrierten Bibliotheken zum Ausführen vom Test nicht betrachtet.

\noindent
Die dritte Leitfrage``Welche Herausforderungen und potenziellen Vorteile ergeben sich bei der Verwendung von Bun 1.0 im Vergleich zu Node.js für Entwickler und Projekte?'' fasst die Erkenntnisse der Arbeit zusammen. Bun bietet das Potential für eine deutlich verbesserte Performance. Dies gilt für die Ausführung von Projekten, Installation von Abhängigkeiten, Minifzieren der Applikation und auch Ausführen von Tests. Darüber hinaus integriert das Executable von Bun direkt Bibliotheken zum Testen, zum Ausführen von TypeScript, zum Verwalten der Abhängigkeiten und zum Bundeln von Applikationen. Das vereinfacht die Entwicklung und Verwaltung von Anwendungen deutlich, da f+r diese grundsätzlichen Tätigkeiten keine Abhängigkeiten gewartet werden müssen. Die Migration kann dabei herausfordernd sein, da Bun viele Frameworks und Pakete noch nicht unterstüzt. Daher empfiehlt es sich für die produktive Nutzung von Bun zu warten, bis das Ökosystem weiterentwickelt und die Unterstützung verbessert hat.


\section{Ausblick} \label{sec:finalThoughts-outlook}
Es gilt die Entwicklung von Bun weiterhin zu beobachten. Die Roadmap für die Entwicklung besitzt noch viele offene Punkte. Zusätzlich kommen durch den offiziellen Release der Version 1.0 viele neue Nutzer dazu, die neue Anwendungsfälle haben oder auch Bugs finden. Inwiefern das Entwicklerteam von Bun dies alles stemmen kann, gilt es zu abzuwarten. \newline
Auf Basis dieser Arbeit können die zwei zentralen Themen Performance und Kompatibilität tiefergehender analysiert werden. Die Evaluation der Performance kann auf die Verwendung von Frameworks ausgeweitet werden. Ist die Performance von Bun mit Express oder Nest.js immer noch höher? Wie viele Ressourcen verwendet Bun beim Ausführen der Frameworks? Darüber hinaus können die Performance-Benchmarks dieser Arbeit in regelmäßigen Zeitabständen wiederholt werden, um die Weiterentwicklung von Bun zu bewerten. Wird die Performance besser oder schlechter?
Zusätzlich kann die Performance von Buns Paketmanager und Bibliothek zum Testen analysiert werden und mit etablierten Lösungen vergliechen werden. Dadurch ist eine komplette Bewertung mit allen integrierten Komponenten in Bun möglich. \newline
In die Untersuchung der Kompatibilität können tatsächlich produktiv verwendete Anwendungen einbezogen werden, die oft komplexere Prozesse und Middleware nutzen. Dadurch können fundiertere Aussagen über Bun als Eins-zu-Eins-Ersatz für Node.js getroffen werden. Darüber hinaus werden neuere Versionen von Bun die Kompatibilität erhöhen und den Migrationsprozess vereinfachen. Sobald hier entscheidende Verbesserungen getroffen werden, gilt es die Betrachtung zu wiederholen.
