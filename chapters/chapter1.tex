% !TeX root = ..\main.tex
\pagestyle{fancy-style}
\npchapter{Einleitung}
\pagenumbering{arabic}

\section{Motivation}
JavaScript ist eine Programmiersprache, die vor allem im Kontext der Web-Entwicklung 
verwendet wird. Aktuell erfreut sich JavaScript großer Beliebtheit. 
In der Umfrage an Entwickler von Stack Overflow wurden mehr als 89.000 Entwickler 
befragt. JavaScript ist zum 11. Jahr in Folge die am häufigsten verwendete Programmiersprache. 
Mehr als 63\% der befragten Entwickler haben JavaScript als beliebteste Technologie 
gewählt. Bei den professionellen Entwicklern ist der Anteil mit mehr als 65\%
sogar noch höher. Außerdem ist TypeScript, eine stark typisierte Programmiersprache, die auf 
JavaScript aufbaut, unter den Teilnehmer auch beliebt. Ca. 39\% aller Entwickler und ca. 44\%
der professionellen Entwickler verwenden auch TypeScript. Damit ist TypeScript die 4. beliebteste 
Programmiersprache. Daraus folgt, dass das Ökosystem von JavaScript eine hohe Praxisrelevanz besitzt. \\

\noindent
JavaScript wird nicht nur für die Entwicklung im Frontend, sondern auch für die Entwicklung im Backend 
verwendet. Denn ungefähr 2\% der weltweit bekannten Server verwenden eine Laufzeitumgebung, die JavaScript 
ausführen kann. Die Laufzeitumgebung wird benötigt, um JavaScript außerhalb des Browsers ausführen zu können.
Hierbei ist Node.js die am weitesten verbreitete Laufzeitumgebung. In einer Umfrage zum Zustand der JavaScript
beantworteten ca. 71\% von 30.000 befragten Entwicklern, dass sie Node.js als Laufzeitumgebung regelmäßig verwenden.
Nur ca. 9\% der befragten Entwickler verwenden Deno und ca. 3\% Bun als eine Alternative zu Node.js. \\

\noindent
Node.js ist der Platzhirsch im Kontext von Laufzeitumgebungen von JavaScript. Dennoch besitzt Node.js Schwächen,
die die Alternativen versuchen zu lösen. Zu den Schwächen zählen eine schwächere Performance bei anspruchsvollen
Aufgaben, die Limitierung auf einen einzelnen Thread und häufige Änderungen an der API.


\section{Zielsetzung}
Zuvor wurde bereits Bun als Alternative zu Node.js genannt. Bun ist eine Laufzeitumgebung, die am 9. September 2023 in der Version 1.0 erschienen ist. Der Entwickler von Bun
haben das Ziel die Nachteile von Node.js zu lösen, damit das eigene Framework immer mehr verwendet wird.
Als Features werden eine hohe Verbesserung der Performance, elegante Schnittstellen und eine angenehme Erfahrung für Entwickler beworben.
Das Ziel dieser Arbeit ist es, Bun als eine Alternative zu Node.js zu evaluieren. Dazu werden die folgenden Leitfragen beantwortet: 
\begin{itemize}
    \item Welche Laufzeitumgebung besitzt die beste Performance?
    \item Ist Bun als Laufzeitumgebung kompatibel mit bestehenden Projekten auf der Basis von Node.js?
\end{itemize}
   
\section{Aufbau der Arbeit}
TODO


