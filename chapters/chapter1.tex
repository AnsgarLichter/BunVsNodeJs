% !TeX root = ..\main.tex
\pagestyle{fancy-style}
\npchapter{Einleitung}
\pagenumbering{arabic}

\section{Motivation}
JavaScript ist eine Programmiersprache, die vor allem im Kontext der Web-Entwicklung verwendet wird. Aktuell erfreut sich JavaScript großer Beliebtheit. In einer Umfrage an Entwickler von Stack Overflow wurden mehr als 89.000 Entwickler befragt. JavaScript ist zum 11. Jahr in Folge die am häufigsten verwendete Programmiersprache. Mehr als 63\% der befragten Entwickler haben JavaScript als beliebteste Technologie gewählt. Bei den professionellen Entwicklern ist der Anteil mit mehr als 65\% sogar noch höher. Außerdem ist TypeScript, eine stark typisierte Programmiersprache, die auf 
JavaScript aufbaut, unter den Teilnehmer auch beliebt. Ca. 39\% aller Entwickler und ca. 44\% der professionellen Entwickler verwenden auch TypeScript. Damit ist TypeScript die 4. beliebteste Programmiersprache. Daraus folgt, dass das Ökosystem von JavaScript eine hohe Praxisrelevanz besitzt. \\

\noindent
JavaScript wird nicht nur für die Entwicklung im Frontend, sondern auch für die Entwicklung im Backend 
verwendet. Denn ungefähr 2\% der weltweit bekannten Server verwenden eine Laufzeitumgebung, die JavaScript 
ausführen kann. Die Laufzeitumgebung wird benötigt, um JavaScript außerhalb des Browsers ausführen zu können.
Hierbei ist Node.js die am weitesten verbreitete Laufzeitumgebung. In einer Umfrage zum Zustand der JavaScript
beantworteten ca. 71\% von 30.000 befragten Entwicklern, dass sie Node.js als Laufzeitumgebung regelmäßig verwenden.
Nur ca. 9\% der befragten Entwickler verwenden Deno und ca. 3\% Bun als eine Alternative zu Node.js. \\

\noindent
Demnach ist Node.js der Platzhirsch im Kontext von Laufzeitumgebungen für JavaScript. Dennoch besitzt Node.js Schwächen, die die Alternativen versuchen zu lösen. Dazu zählen eine schwächere Performance bei anspruchsvollen
Aufgaben, die Limitierung auf einen einzelnen Thread und häufige Änderungen an der API.


\section{Zielsetzung}
Zuvor wurde Bun als eine mögliche Alternative zu Node.js erwähnt. Bun ist eine Laufzeitumgebung, die am 9. September 2023 in der Version 1.0 veröffentlicht worden ist. Die Entwickler von Bun haben das Ziel gesetzt, die Herausforderungen von Node.js zu bewältigen, um so die Akzeptanz des eigenen Frameworks zu steigern. Sie werben mit Features wie erheblicher Performancesteigerung, eleganten Schnittstellen und einer angenehmen Entwicklererfahrung. \\

\noindent
Das Ziel der Arbeit besteht darin, Bun als eine mögliche Alternative zu Node.js zu evaluieren. Dabei liegt der Fokus
auf der Überprüfung, ob die beworbenen Features tatsächlich in die Praxis umgesetzt wurden. Falls die Performance von Bun besser ist als die von Node.js, muss die Kompatibilität zu bestehenden Projekte geprüft werden. Andernfalls steigt der Migrationsaufwand enorm an und beeinflusst die Akzeptanz des Frameworks negativ. Zu diesem Zweck werden die folgenden Leitfragen herangezogen:
\begin{itemize}
    \item Welche Laufzeitumgebung bietet die beste Performance?
    \item Ist Bun als Laufzeitumgebung kompatibel mit bestehenden Projekten auf der Basis von Node.js?
\end{itemize}

\section{Aufbau der Arbeit}
TODO


